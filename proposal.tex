\documentclass[11pt,letterpaper,oneside]{article}
\usepackage{url}
\usepackage{fullpage}
%\setlength{\columnsep}{1.7em}

\begin{document}
	\thispagestyle{empty}
	\title{Speakurity: Using Speaker Recognition for Biometric Security}
	\author{Chris Li, Grayson Smith, Carey Zhang}
	\date{\today}
	
	%\twocolumn[%
	% \centerline{\Large \bf Speakurity: Biometric Security with Speaker Recognition} %% Paper title
  %
	% \medskip
  %
	% \centerline{\bf Chris Li, Grayson Smith, Carey Zhang}
	% \centerline{\bf EE201 Final Lab}
	%      %% Author name(s)
	% \vspace{2.5em}
	% ]
	\centerline{\Large \bf Speakurity: Biometric Security with Speaker Recognition} %% Paper title
  
	\medskip
  
	\centerline{\bf Chris Li, Grayson Smith, Carey Zhang}
	\centerline{\bf EE201 Final Lab}
	
	\section{Introduction}
	Biometrics takes advantage of unique human traits such as fingerprints or speech and applies it to access control.  
	There is global demand, especially from governments, for biometric security systems.  Speech is an ideal biometric trait, 
	as it is fairly unique and most people are capable of producing it.  Potential applications include automated authentication
	of a user over the telephone or internet.
	
	
	There are various metrics used in determining the robustness of a biometric security system\cite{security}.  The following are of interest to us.
	\begin{enumerate}
		\item Accuracy - How often does the system recognize an authorized speaker?
		\item False Accept Rate - How often does the system recognize an unauthorized speaker?
		\item Template Capacity - How many unique voices can be stored before the False Accept Rate goes too high?
	\end{enumerate}

	\section{Outline}
	We will have a Mealy state machine with several states including Initial, Learn, Learning, and Recognize.  We will use the following inputs and outputs:
	\begin{enumerate}
		\item Sw0: Toggle between Learn and Recognize
		\item Sw1-7: Go from Learn to Learning, where the system will store inputs from the FFT.
		\item Ld0: Turns on when learning
		\item Ld1-7: Turns on when its corresponding speaker is recognized.
		\item SSD0-3: Shows the current state: I, L, Ln (where n is a number 1-7), A (Authorized), U (Unauthorized/unrecognized)
	\end{enumerate}
	
	\section{Plan}
	We will be using the Nexys 2 board to synthesize our Verilog.  In addition, we will need a microphone with digital outputs or an A to D converter.
	
	First we will prototype the system using MATLAB, determining the necessary parameters for a robust system.  
	Next, we will implement the Cooley-Tukey FFT algorithm using a fixed number of points.  
	Finally, we will convert our MATLAB code to a hardware implementation in Verilog.
		
	\begin{thebibliography}{9}
		\bibitem{security}
		\url{http://www.ccert.edu.cn/education/cissp/hism/039-041.html}
		
	\end{thebibliography}
\end{document}