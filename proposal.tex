\documentclass[11pt,letterpaper,oneside]{article}
\usepackage{url}
\usepackage{fullpage}
%\setlength{\columnsep}{1.7em}

\begin{document}
	\thispagestyle{empty}
	\title{Speakurity: Using Speaker Recognition for Biometric Security}
	\author{Chris Li, Grayson Smith, Carey Zhang}
	\date{\today}
	
	%\twocolumn[%
	% \centerline{\Large \bf Speakurity: Biometric Security with Speaker Recognition} %% Paper title
  %
	% \medskip
  %
	% \centerline{\bf Chris Li, Grayson Smith, Carey Zhang}
	% \centerline{\bf EE201 Final Lab}
	%      %% Author name(s)
	% \vspace{2.5em}
	% ]
	\centerline{\Large \bf Veritune: Auto-Tuning in Verilog} %% Paper title
  
	\medskip
  
	\centerline{\bf Chris Li, Grayson Smith, Carey Zhang}
	\centerline{\bf EE201 Final Lab}
	
	\section{Introduction}
	Though it used to be an embarrassing secret for the music industry, the Auto-Tune effect has become a staple of modern music \cite{autotune}. 
	When used sparingly, it can correct a slightly off-key singer; 
	when abused, it can produce a unique audio effect.  
	At its heart, the effect uses a phase vocoder to scale the signal in the frequency domain.
	
	We will develop a device that can record and playback audio with pitch shifted by an amount specified by the user.
	If time and processing power permit, we will make the device operate in realtime.
	\section{Outline}
	We will have a Moore state machine with several states including Initial, Listen, Compute, Idle, Play.  We will use the following inputs and outputs:
	\begin{enumerate}
		\item Sw7: Toggles shift up/shift down
		\item Sw6-0: Specifies the amount of shift as a binary number
		\item Btn3: Reset/Clear
		\item Btn0: Record $\rightarrow$ Play
		\item Btn1: Stop
		\item SSD0-3: Shows the decimal amount of phase shift.  SSD0 shows the sign.
	\end{enumerate}
	
	\section{Plan}
	We will be using the Nexys 2 board to synthesize our Verilog.  In addition, we will need a microphone with digital outputs as well as a speaker.
	
	First we will prototype the system using \textsc{matlab}, determining the necessary parameters for a robust system.  
	Next, we will implement the Cooley-Tukey FFT algorithm using 1024 points.
	Next, we will implement the phase vocoder in \textsc{matlab}.
	Finally, we will convert our \textsc{matlab} code to a hardware implementation in Verilog.
		
	\begin{thebibliography}{9}
		\bibitem{autotune}
		\url{http://www.time.com/time/magazine/article/0,9171,1877372-2,00.html}
		
	\end{thebibliography}
\end{document}